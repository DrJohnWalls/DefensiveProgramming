% !TEX options=--shell-escape

\documentclass[11pt]{beamer}
\usetheme[sectionpage=none, numbering=none]{metropolis}           % Use metropolis theme
    % Kill page numbers with [numbering=none] in next line:
    % To do printouts, add ", handout"  after aspectratio.
\usepackage{makecell}
\usepackage{booktabs}
%\usepackage{graphicx}
\usepackage{color}
\usepackage{minted}


\title{Defensive Programming}
\author{    \small Nick Eubank \\
            \scriptsize{CSDI, Vanderbilt University}\vspace*{.15in}}
\date{\vspace*{.3in} \today}


% This is the beginning of a real document!
\begin{document}


\begin{frame}
    \maketitle
\end{frame}


\begin{frame}[c]{Goals}
\begin{enumerate}
    \pause \item Introduce principles of \emph{Defensive Programming}\\\vspace{0.1cm}
    \pause \item Learn four specific ``best practices''
    \begin{itemize}
        \item Use tests
        \item Don't duplicate information
        \item Don't transcribe, export
        \item Use good style
    \end{itemize}
\end{enumerate}
\end{frame}

\begin{frame}[c]{Defensive programming}
\pause Philosophy of writing code \pause motivated by the simple proposition: \\

\begin{center}
    \pause \alert{People are bad at writing code}
\end{center}
\pause If we want to avoid errors, \pause not enough to \alert{``just be careful.''}\\
\pause \vspace{0.2cm}\\$\Rightarrow$ Need strategies take take our fallibility into account
\end{frame}


\begin{frame}[c]{Defensive programming}
    Set of best practices designed to:
    \begin{enumerate}
        \pause \item Minimize opportunities for errors to enter code
        \pause \item Maximize the probability that \emph{when} we commit errors, we catch them quickly
    \end{enumerate}
\end{frame}


\begin{frame}\frametitle{Do I need defensive programming?}
\begin{center}
    \pause \LARGE \alert{YES.}
\end{center}
\end{frame}

\begin{frame}\frametitle{Do I need defensive programming?}
\begin{center}
    \alert{``To Err is Human''}
\end{center}
\begin{itemize}
    \pause \item \emph{Among professional programmers}, average error rate is 10 - 50 bugs per 1,000 lines of delivered code \\
    {\color{gray}{Steve McConnell, 1993}}
\end{itemize}
``Bugs'' $\nRightarrow$ syntax errors
\end{frame}

\begin{frame}\frametitle{Do I need defensive programming?}
QJPS Replication Review:
\begin{itemize}
    \item Before publication, test whether replication packages run and generate results in the paper.
\end{itemize}

\pause From 2012 - 2016:
\begin{itemize}
    \pause \item 4 packages passed without modifications
    \pause \item \alert{\emph{58\%}} of packages generated results that were different from those in the paper.
\end{itemize}
\end{frame}

\begin{frame}[t]\frametitle{Do I need defensive programming?}
    \begin{enumerate}
        \pause \item If a correlation exists between sophistication of analysis and likelihood of errors, it is if anything positive.
        \begin{itemize}
            \item Senior, junior, fancy, basic: all had problems!
        \end{itemize}
        \pause \item Even if you trust yourself, do you trust your coauthors?
        \begin{itemize}
            \pause \item (Do you trust the version of you that wrote that code at 3am?))
        \end{itemize}
        \pause \item Do you trust the people who made the dataset you're using?
        \begin{itemize}
            \pause \item If you estimate the share of a population that's female, and someone left a \texttt{7} in the \texttt{female} variable, if you don't catch it, that means your answer is \emph{wrong}.
        \end{itemize}
    \end{enumerate}
\end{frame}

\begin{frame}[c]{Defensive programming}
    Set of best practices designed to:
    \begin{enumerate}
        \item Minimize opportunities for errors to enter code
        \item Maximize the probability that \emph{when} we commit errors, we catch them quickly
    \end{enumerate}
\end{frame}



\begin{frame}[c]{Four Skills}
    \tableofcontents
\end{frame}



\section{Write tests}

\begin{frame}[c]{Four Skills}
    \tableofcontents[current]
\end{frame}

\begin{frame}[c, fragile]{Tests}
Lines of code that assert something about the data
\begin{itemize}
    \item Evaluate to \texttt{True} or \texttt{False}
\end{itemize}
\vspace{0.5cm}\\
\pause e.g.
\begin{minted}{r}
df = read.csv('state_populations.csv')
stopifnot( nrow(df) == 50 )
\end{minted}
\end{frame}

\begin{frame}[c]{But I ``test'' informally...}
Value of tests:
    \begin{enumerate}
        \pause \item Explicit form of checking data is doing what you expect
        \pause \item Unlike just looking at result interactively, executes \alert{every time you run your code}
        \begin{itemize}
            \pause \item If you or co-author change upstream data or code and introduce a mistake, tests will catch.
        \end{itemize}
    \end{enumerate}
\end{frame}


\begin{frame}[fragile]\frametitle{Writing tests: R}
Is age always positive?

\vspace{0.5cm}
This will pass (do nothing):
\begin{minted}{r}
age = c(42, 20, 31, 18)
# Make sure age is positive:
stopifnot( age > 0 )
\end{minted}
\pause
\vspace{1cm}
But if, for example, ``missing'' was coded as -99, this would throw an error:
\begin{minted}{r}
age = c(42, 20, 31, -99)
stopifnot( age > 0 )
\end{minted}
\end{frame}

\begin{frame}[fragile]\frametitle{Writing tests: R}
For vectors, \texttt{stopifnot} checks if ALL values are TRUE.
\vspace{1cm}

This will fail:
\begin{minted}{r}
# Are all values True?
v = c(1, 2, 3)
stopifnot( v == 2 )
\end{minted}
\vspace{1cm}
This will pass:
\begin{minted}{r}
stopifnot( v > 0 )
\end{minted}
\end{frame}

\begin{frame}[fragile]\frametitle{Writing tests: R}
If you want to see if test holds for at least \emph{some} observations, use \texttt{any}.
\vspace{1cm}

\begin{minted}{r}
# Are there at least some values that are 2?
v = c(1, 2, 3)
stopifnot( any(v == 2) )
\end{minted}
This will pass.
\end{frame}

\begin{frame}[fragile]\frametitle{Writing tests}

Can combine with functions:
\vspace{1cm}
\begin{minted}{r}
stopifnot( length(VECTOR) == 100 )
\end{minted}
\end{frame}


\begin{frame}[fragile]\frametitle{Writing tests: Stata}
Is age always positive?

\vspace{0.5cm}
\begin{minted}{stata}
assert age > 0
\end{minted}
\pause
50 States in data?
\begin{minted}{stata}
count
assert r(N) == 50
\end{minted}
\end{frame}


\begin{frame}\frametitle{How to write tests}
    Go to:
\end{frame}




\begin{frame}\frametitle{Tests: Exercise 1}
Setup:
\begin{enumerate}
    \pause \item Download \texttt{exercises} folder
    \pause \item Open \texttt{starter\_code.R} or \texttt{starter\_code.do}, set the working directory to the \texttt{exercises} folder.
\end{enumerate}

Exercises:
\begin{enumerate}
    \pause \item The World Development Indicator (WDI) data has duplicate entries. \alert{Write a test that fails because there shouldn't be duplicates.}
    \pause \item The countries in Polity should be a perfect subset of the countries in the WDI dataset, but they are not. \alert{Write a test that fails because of this.}
\end{enumerate}
\end{frame}


\begin{frame}\frametitle{Tests: When to write them}

\begin{itemize}
    \pause \item \textbf{After merges} No where are problems with data made more clear then in a merge. ALWAYS add tests after a merge!
    \pause \item \textbf{After complicated manipulations} If you had to think about it, you should test to do it.
    \pause \item \textbf{Before dropping observations}
\end{itemize}

\vspace{1cm}
\pause Most of use check things interactively to make sure we did it right. A good rule of thumb is that when you catch yourself checking something interactively, stop and write it as a test.

\end{frame}


\section{Don't duplicate information}


\begin{frame}[c]{Four Skills}
    \tableofcontents[currentsection]
\end{frame}


\begin{frame}[fragile]{Don't duplicate information}

\begin{itemize}
    \pause \item If information is represented in many places, when you make changes you have to find all those places.
    \pause \item If information is represented once, and everything else points back to that representation, one change will \emph{always} change everything.
\end{itemize}

\end{frame}

\begin{frame}[fragile]\frametitle{Don't duplicate information}

Duplicated information:
\vspace{0.5cm}
\begin{minted}{r}
df$var1 <- gsub("armadillo", "Mr. Armadillo",
                df$var1)
df$var2 <- gsub("armadillo", "Mr. Armadillo",
                df$var2)

...
[Other manipulations]
...

df$var7 <- gsub("armadillo", "Mr. Armadillo",
                df$var7)
\end{minted}
\end{frame}

\begin{frame}[fragile]\frametitle{Don't duplicate information}
One representation:
\vspace{0.5cm}
\begin{minted}{r}
pre_change_name <- "armadillo"
replacement_name <- "Mr. Armadillo"

df$var1 <- gsub(pre_change_name,
                replacement_name, df$var1)
df$var2 <- gsub(pre_change_name,
                replacement_name, df$var2)
df$var7 <- gsub(pre_change_name,
                replacement_name, df$var7)
\end{minted}
\end{frame}


\begin{frame}[fragile]\frametitle{Don't duplicate information}
One representation:
\vspace{1cm}
\begin{minted}{r}
pre_change_name <- "armadillo"
replacement_name <- "Dr. Armadillo"

df$var1 <- gsub(pre_change_name,
                replacement_name, df$var1)
df$var2 <- gsub(pre_change_name,
                replacement_name, df$var2)
df$var7 <- gsub(pre_change_name,
                replacement_name, df$var7)
\end{minted}
\end{frame}


\begin{frame}[fragile]\frametitle{Don't duplicate information}
\begin{minted}{stata}
replace var1 = "Mr. Armadillo" ///
               if var1 == "armadillo"
replace var2 = "Mr. Armadillo" ///
               if var2 == "armadillo"
replace var7 = "Mr. Armadillo" ///
               if var3 == "armadillo"
\end{minted}
\end{frame}

\begin{frame}[fragile]\frametitle{Don't duplicate information}
\begin{minted}{stata}
local pre_change_name = "armadillo"
local replacement_name = "Mr. Armadillo"
replace var1 = "`replacement_name'" ///
               if var1 == "`pre_change_name'"
replace var2 = "`replacement_name'" ///
               if var2 == "`pre_change_name'"
replace var7 = "`replacement_name'" ///
               if var3 == "`pre_change_name'"
\end{minted}
\pause (Or write as loop)
\end{frame}



\begin{frame}\frametitle{Don't Duplicate: Exercise 2}
Exercises:
\begin{itemize}
    \item All the regressions in the code you have use the same base specification. Consolidate the representation of that base specification.
    \item Now add \texttt{population} as a control to all you regressions. You should only have to add it once!
\end{itemize}
\end{frame}

\section{Don't transcribe, export}

\begin{frame}[c]{Four Skills}
    \tableofcontents[currentsection]
\end{frame}



\begin{frame}\frametitle{Don't transcribe results!}

Number one reason papers don't match real results at QJPS.
\vspace{0.5cm}

R:
\begin{itemize}
    \item \texttt{stargazer}
    \item Tutorial: \url{http://jakeruss.com/cheatsheets/stargazer.html}
    \item Custom: \url{http://stanford.edu/~ejdemyr/r-tutorials/tables-in-r/}
\end{itemize}

Stata:
\begin{itemize}
    \item Summary: \url{http://www.nickeubank.com/exporting-results-stata-latex/}
\end{itemize}

\textbf{Use for numbers in your text as well!}
\end{frame}

\begin{frame}[fragile]\frametitle{Don't transcribe results: Exercise 3}
Oh man, our tests found all these problems. Ugh, why did we put all those old results in by hand?! Now we have to copy them again. Or... we could make the automatically updating!

1) Export the regression table at the end of \texttt{starter\_code.R} using stargazer.

2) Open our latex analysis file (\texttt{start\_latex.tex}).

3) Import it into your latex document using the \texttt{\\input\{\}} command.
\end{frame}

\begin{frame}[c, fragile]{Don't transcribe results: Exercise 3}
Make tables:
\begin{minted}{r}
stargazer(list(model1, model2, model3, model4),
          title="TITLE HERE",
          type="latex",
          out='FILE_NAME.tex' )
\end{minted}
\hline
\begin{minted}{r}
eststo clear
eststo: reg polity gdp_per_cap
eststo: reg polity gdp_per_cap under5_mortality
esttab using your_file_name.tex, replace ///
       title("My regressions!")
\end{minted}
\hline
Latex Import: \texttt{\\input\{your\_file\_name.tex\}}
\end{frame}


\begin{frame}[fragile]\frametitle{Don't transcribe: Exercise 2 (Part 2)}
Now, in the text, we say that households have an average size of 8, but we know that's wrong. Can you export the average size of the household from R and import it into LaTeX? Hint: Here's how you write a number to a file as text:

\begin{minted}{r}
    x = 1/3
    x_as_string = format(x, digits=2)
    write(x_as_string, "my_file.tex")
\end{minted}
\end{frame}



\section{Use good style}
\begin{frame}[c]{Four Skills}
    \tableofcontents[currentsection]
\end{frame}

\begin{frame}[c]{Style}
Style isn't just about aesthetics; it's about making code \alert{readable} so errors are easy to see.
\begin{itemize}
    \item Whitespace is free; use it.
    \item Use informative variable names (not x, y)
    \item COMMENT
\end{itemize}
\end{frame}

\begin{frame}[c, fragile]{Style}
Bad:
\begin{minted}{r}
df=read.csv('file_90823409.csv')
df$var07=df$var07+11
df=df[df$var07>65]
lm("pid~var09")
\end{minted}
\end{frame}

\begin{frame}[c, fragile]{Style}
    \begin{minted}{r}
    # Load voter survey data from 2007
    voters = read.csv('file_90823409.csv')

    # var07 is age in 2007. Want age today (2018)
    voters$age = voters$var07 + 11

    # Let's see how income predicts voterid for
    # people over 65 today, so subset to
    # people over 65 today and regress.

    voters = voters[df$age>65]
    voters$income = df$var09

    lm("pid ~ income")
    \end{minted}
\end{frame}

\begin{frame}[c]{Thanks!}
    Resources:
    \url{www.nickeubank.com/replication}
\end{frame}
\end{document}
